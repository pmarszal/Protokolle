Die Durchführung des Versuches kann grob in vier Abschnitte unterteilt werden.
Im ersten Teil wird die Form der PSF der beiden Laser bestimmt.
Dazu werden 80~nm große Goldkügelchen als Probe verwendet.
Da die PSF der reinen fokussierten Laser betrachtet werden soll, wird die Phasenplatte aus dem Strahlengang entfernt.

Der Strahlteiler, der einen Teil des Lichts an den Photomultiplier (PMT) weitergibt wird eingesetzt.
Mithilfe der Kamera wird die Fokusebene eingestellt.
Dazu wird die z-Achse zunächst grob per Hand in die Nähe der Goldkügelchen bewegt, und anschließend die Feineinstellung mit dem Motor durchgeführt.

Nun können die Goldkügelchen mit dem Imspector im Scanmodus betrachtet werden.
Für die Messung wird nun ein Goldkügelchen möglichst zentriert und sein Bild in der $xy$-, $xz$- und $yz$-Ebene gescannt. 
Aus diesen Bildern lässt sich nachher die PSF rekonstruieren.

Als nächstes wird die Tiefendiskriminierung des Anregungslasers bestimmt.
Hierfür wird ein dünner Farbstofffilm als Probe verwendet.
Anstelle des PMT wird nun eine Avalanche-Photodiode (APD) als Detektor verwendet.
Die APD is direkt an eine Faser gekoppelt deren Kern die Rolle der konfokalen Lochblende übernimmt.

Der Farbstofffilm wird nun in z-Richtung gescannt und die gemessene Intensität in abhängigkeit von der z-Tiefe bestimmt.

Für den zweiten Teil wird die Überlagerung von Anregungslaser und STED-Laser untersucht.
Dazu müssen zunächst die PSF der beiden Laser mittels des dikroitischen Spiegels überlagert werden.
Es empfiehlt sich dies direkt an die Messung der Goldkügelchen anzuschließen und die durch ein Goldkügelchen erzeugten PSFs zu überlagern.

Jetzt kann die Auslöschung der Fluoreszenz durch den STED-Laser untersucht werden. 
Als Probe wird nun eine Lösung von fluoreszierenden Nanoteilchen verwendet (Größe 40~nm).

Bei konstanter Anregungsleistung wird die gemessene Intensität in Abhängigkeit von der STED-Leistung bestimmt.
Pro STED-Leistung wird sechs mal die integrierte Intensität über einen Bereich bestimmt. Zwei mal nur mit dem Anregungslaser, zweimal mit Anregungs- und STED-Laser und wieder zweimal nur mit dem Anregungslaser.
Die Messungen mit dem Anregungslaser, werden vor und nach der STED-Beleuchtung durchgeführt um Effekte durch Ausbleichen der Probe auszugleichen.

Dies wird für steigende STED-Leistungen durchgeführt. Wichtig für die Auswertung sind die Verhältnisse zwischen reiner Fluoreszenz durch den Anregungslaser und Auslöschung durch den STED-Laser.

Im dritten Teil des Versuches wird die Auflösung der STED-Mikroskopie bestimmt.
Vorbereitend wird dafür mittels der Phasenplatte die 'Doughnut'-Form des STED-Lasers erzeugt. 
Um die PSF möglichst gut justieren zu können wird wieder die Goldkügelchenprobe verwendet, und die Form der STED-PSF durch Verschieben der Phasenplatte justiert.
Außerdem muss, das Minimum der STED-PSF mit der Anregungs-PSF überlagert werden.
Für die Auswertung wird die Restintensität am Minimum der STED-PSF bestimmt.

Die Auflösung der STED-Mikroskopie wird nun anhand der fluoreszierenden Probe aus dem zweiten Teil bestimmt.

Für steigende STED-Intensitäten werden Bilder der fluoresizierenden Kügelchen aufgenommen. 
Die Probe wird dabei nach jeder Aufnahme neu justiert, damit das Ausbleichen der Fluoreszenzpartikel möglichst wenig Einfluss auf die Messungen nimmt.

Im vierten Teil des Versuchs werden nun Säugetier-Zellen mithilfe von STED-MIkroskopie untersucht.
Bei den Fluoreszierenden Teilen der Zelle handelt es sich um von fluoreszierenden Antikörpern besetztes Tubulin.

Zunächst wird eine Stelle mit dichten Tubulinstrukturen nur mithilfe des Anregungslasers gesucht.
Von dieser Stelle werden nun Aufnahmen nur mit dem Anregungslaser, mit Anregungs- und STED-Laser und nur mit dem STED-Laser gemacht.




