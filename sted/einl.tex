Bis zum 16. Jahrhundert war die reale Welt eine Welt des Meters. 
Die meisten damals bekannten Lebewesen haben sich auf Skalen des Meters bewegt, seien es millimetergroße Insekten oder meterlange Wale.
\\
Im 16. Jahrhundert aber änderte eine Erfindung die Sicht auf die Welt. 
Das Mikroskop ermöglichte die Erkundung der kleinsten Längen. 
Es ermöglichte die Entdeckung von Zellen, Bakterien und auch physikalischen Phänomenen wie der Brown'schen Molekularbewegung.
\\
Seitdem unterliegt das Prinzip der Mikroskopie einem steten Wandel, um noch kleinere Skalen untersuchen zu können.
Eine grundlegende physikalische Begrenzung jedoch stellt die Wellenoptik dar.
Beugung beschränkt die Auflösung der Mikroskopie.
\\
Diese Grenze kann jedoch mithilfe der STED-Mikroskopie umgangen werden. 
\\
Dieser Versuch beschäftigt sich mit der Untersuchung und Funktionsweise der STED-Mikroskopie. Er wird deutlich die Vorteile gegenüber herkömmlicher Mikroskopietechniken aufzeigen.
