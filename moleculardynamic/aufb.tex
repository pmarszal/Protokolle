\subsection{Molekulardynamik Simulation}
Die numerische Behandlung von Molekülen basiert zunächst auf der quantenmechanischen Beschreibung von Molekülen.
Allerdings stößt die Berechnung von quantenmechanischen Systemen bereits ab einer Zahl von 10 Atomen an ihre Grenzen.
Um dennoch sinnvolle numerische Ergebnisse in endlicher Zeit zu erhalten, werden für die Molekulardynamik Simulationen eine Reihe von Approximationen eingesetzt.
So wird die Bewegung der Elektronen eines Atoms, von der Bewegung des Kerns, entkoppelt, indem angenommen wird, dass der Atomkern sich im Vergleich zu den Elektronen sehr langsam bewegt.
Damit kann die Bewegung der Elektronen als die Bewegung in einem ruhenden Potential der Kerne betrachtet werden.
Der Einfluss der Elektronen auf den Atomkern wird wiederum durch ein durch die Elektronen verursachtes effektives Potential beschrieben.
\\ \noindent
Dieses effektive Potential wird durch die Koordinaten des Atomkerns bestimmt. Es berücksichtigt sowohl Bindungen in einem Molekül als auch langreichweitige Wechselwirkungen wie die van-der-Waals-Kräfte.
\\ \noindent
Die Bewegung der Atomkerne kann darauf hin nach den klassischen Bewegungsgleichungen integriert werden.
Hier wird der \emph{leap frog} Algorithmus zur Integration verwendet, wie er ausführlich in \cite{bishop2006pattern} behandelt wird:
\begin{align}
v\left(t+\frac{\Delta t}{2}\right) = v\left(t-\frac{\Delta t}{2}\right) + \frac{F(t)}{m}\Delta t, \\
r\left(t+\frac{\Delta t}{2}\right) = r(t) + v\left(t+\frac{\Delta t}{2}\right)\Delta t.
\end{align}
\\ \noindent
Des weiteren müssen für eine sinnvolle Simulation einige technische Grenzen gesetzt werden.
Die Integrationsschrittweite wird den schnellsten erwarteten Fluktuationen entsprechend gewählt. In diesem Versuch sind dies Fluktuationen von Wasserstoffatomen, die auf einer Zeitskala von $10^{-14}$~s geschehen. Als Schrittweite wird deswegen $\Delta t = 10^{-15}$~s gewählt.
Um Randeffekte zu vermeiden werden periodische Randbedingungen gewählt, wodurch prinzipiell ein unendlich großes System simuliert wird.
\\ \noindent
Die Molekulardynamik Simulation wird in diesem Versuch mit dem Softwarepaket \emph{GROMACS} \cite{gromacs} durchgeführt.

%\begin{table}
%\centering
%\caption{}
%\begin{tabular}
%Wärmebadkopplung Berendsen
%Zeitkonstante 1~ps
%Proteinkraftfeld AMBER99SB protein, nucleic AMBER94 \cite{PROT}
%\end{tabular}
%\end{table}

\subsection{Principal Component Analysis}
Die Hauptkomponentenanalyse oder PCA (engl. Principal Component Analysis) ist eine Methode der Datenanalyse, die aus einem Datensatz Linearkombinationen von Variablen erzeugt die maximalen Einfluss auf die Dynamik haben.
Dazu wird zunächst die Kovarianz-Matrix des Datensatzes bestimmt. Anschließend wird diese diagonalisiert und die Diagonalelemente ihrer Größe nach geordnet. Nun werden die Daten in die Basis mit diagonaler Kovarianz-Matrix transformiert.
Die Basisvektoren dieser Basis entsprechen den Hauptkomponenten oder PCs (engl. Principal Components).
Das zugehörige Diagonalelement der Kovarianz-Matrix entspricht der auf dieser Achse auftretender Varianz der Daten.
\\ \noindent
Die Hauptkomponenten mit der größten Varianz entsprechen den Variablen mit der stärksten Aussagekraft über den Zustand des Systems. Vernachlässigt man Hauptkomponenten geringer Varianz, so lässt sich die Dimensionalität des Datensatzes stark verringern ohne Information einzubüßen \cite{bishop2006pattern}.\\
In diesem Versuch wird es hilfreich sein komplexe Proteinbewegungen auf zwei essentielle Bewegungen zu reduzieren um die Arbeitsweise des Proteins zu untersuchen.
