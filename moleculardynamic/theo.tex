\subsection{Van-der-Waals Gas}
Beim van-der-Waals Gas handelt es sich um eine Erweiterung des idealen Gases.
Im Gegensatz zum idealen Gas wird angenommen, dass die Teilchen des Gases eine endliche Ausdehnung haben und eine zwischen den Teilchen wirkende anziehende Kraft wirkt.
Aus diesen Annahmen lässt sich eine Modifikation der idealen Zustandsgleichung, die van-der-Waals-Gleichung \ref{eq:vdw}, herleiten \cite{Dem1}:
\begin{align}
\left( p+\frac{a\cdot N^2}{V^2}\right)\left(\frac{V}{N}-b\right) = k_B T \label{eq:vdw}
\end{align}
Hierbei bezeichnet $p$ den Druck, $N$ die Teilchenzahl, $V$ das Volumen, $k_B$ die Boltzmann-Konstante und $T$ die Temperatur.
Die beiden Parameter $a$ und $b$ sind stoffabhängig und können experimentell bestimmt werden.

\subsection{Proteine}
Proteine gehören zu den Grundbausteinen von lebenden Organismen. Sie bestehen aus Ketten von Aminosäuren.
Der Aufbau von Proteinen lässt sich in drei bis vier unterschiedliche Kategorien aufteilen.
Die grundlegendste Struktur von Proteinen, die Primärstruktur, bezeichnet die Abfolge von Aminosäuren im Protein. In der Biologie kommen 20 verschiedene Aminosäuren vor.
Die Reihenfolge der Aminosäuren bestimmt durch Anziehungskräfte der Aminorestgruppen zu einer charakteristischen Faltung des Proteins.
Dabei bestimmen hauptsächlich Wasserstoffbrückenbindungen die Struktur, allerdings gibt es Aminosäuren wie Cystein, das kovalente Bindungen mit anderen Cystein-Molekülen eingehen kann \cite{Campbell}.
\\ \noindent
Die Faltung des Proteins führt auf der nächsten Ebene zur Ausbildung der Sekundärstruktur. Die zwei am häufigsten anzutreffenden Sekundärstrukturen sind die $\alpha$-Helix und die $\beta$-Faltblatt Struktur.
\\ \noindent
Eine weitere Ebene darüber ist die Tertiärstruktur. Sie beschreibt die Anordnung der Sekundärstrukturen im dreidimensionalen Raum. Für Proteine, die nur aus einer Aminosäurekette bestehen, beschreibt sie die endgültige Form des Proteins\cite{Campbell}.
\\ \noindent
Viele Proteine jedoch bestehen aus mehreren Aminosäureketten. Die Anordnung der Aminosäureketten zusammen wird durch die Quartärstruktur beschrieben.
