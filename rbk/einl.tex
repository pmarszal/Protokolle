Konvektion durch Temperaturunterschiede ist mit Abstand die häufigste Art von Fluidbewegung im Universum. Sie tritt in Sternen, Atmosphären und Planetenkernen auf.
Aber die Untersuchung der Konvektion beschränkt sich nicht nur auf diese Phänomene. Modellbasierte Realisierungen von Konvektion erlauben Rückschluss auf chaotische Dynamik und Komplexität von hydrodynamischen Prozessen.
\\
Ein oft herangezogenes Modell für die Konvektion von von unten beheizten Flüssigkeiten ist die Rayleigh-B\'enard-Konvektion. Sie veranschaulicht unter Anderem, die spontane Ausbildung von makroskopischen Strömungsmustern.
\\
In diesem Versuch wird die Rayleigh-B\'enard-Konvektion sowohl numerisch als auch experimentell untersucht. 
