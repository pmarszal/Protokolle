\section{Durchführung}
Das Phänomen der Rayleigh-Benard-Konvektion wird sowohl numerisch als auch experimentell untersucht.
\subsection{Simulation}
Mithilfe des Programms \emph{Comsol Multiphysics} wird die Rayleigh-Benard-Konvektion in einer zweidimensionalen Box simuliert. 
Die Simulation wird für die Rayleigh-Zahlen $10^3, 10^4, 10^5$ und $10^6$ durchgeführt. 
Ausgegeben werden dabei die Temperatur- und Geschwindigkeitsprofile entlang der Achse durch die Mitte der Box.
\\
Zudem wird die Nusselt-Zahl als das Integral des Wärmegradienten über die Grenzfläche der Box bestimmt.

\subsection{Experimentelle Betrachtung}
\subsubsection{Schattenwurfmethode}
Zunächst wird die Dynamik des Fluids im Versuchsaufbau, visuell beschrieben.
Dazu wird die Glasbox so beleuchtet, dass auf einer Seite durch die Plumes erzeugte Schatten zu erkennen sind.
Die Schatten entstehen durch die Brechung des Lichts an starken Dichtegradienten des Fluids, die Folge von Temperaturunterschieden sind.
\\
Die Geschwindigkeit der Plumes wird mithilfe einer Stoppuhr gemessen. 
Es werden sowohl auf einer Seite aufsteigende warme Plumes, als auch auf der anderen Seite fallende kalte Plumes gemessen.
\\
Die Strecke über die gemessen wird beträgt ca. 2.5~cm. Für jede Seite werden 10 Messwerte gesammelt.
\subsubsection{Beweglicher Thermistor}
Das Temperatur- und Geschwindigkeitsprofil des Fluids wird nun mit einem einzelnen Thermistor bestimmt.
Der Thermistor wird zunächst direkt an der warmen Platte in der Mitte der Box positioniert. 
Mit einem bereitgestellten Messprogramm werden nun 2048 Messwerte der Temperatur mit einer Abtastfrequenz von 11.5~Hz auf genommen. 
\\ 
Dann wird der Thermistor um 1~cm nach oben verschoben, und eine Messung für die neue Höhe gestartet. Dies wird bis zu einer Höhe von 10~cm durchgeführt.
\\
Für die ersten 1.5~cm werden die Messungen alle 0.1~cm durchgeführt. 
Für die Bestimmung der thermischen Grenzschicht wird zudem eine noch höhere Auflösung benötigt, weswegen zusätzliche Messwerte bei 0.05, 0.15, 0.25 und 0.35~cm aufgenommen werden, die allerdings nur aus 512 Messpunkten bestehen.

\subsubsection{Thermistorarray}
Anschließend wird eine lange Messung mit einem Thermistorarray aus sechs Thermistoren durchgeführt.
Mit allen Thermistoren wird über 30~Stunden hinweg die Temperatur gemessen.
Aus der zeitlichen Korrelation der Messwerte der einzelnen Thermistoren lässt sich die Geschwindigkeit von Plumes bestimmen.

