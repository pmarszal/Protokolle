\section{Aufbau}
Der Versuchsaufbau besteht aus einer \SI{20}{\centi\meter} hohen Box, gefüllt mit Wasser. 
Am Boden der Zelle befindet sich eine Heizplatte, die eine Temperatur von ca. \SI{20}{\celsius} besitzt.
Nach oben hin wird die Zelle durch eine ca. \SI{10}{\celsius} kalte Kühlplatte begrenzt. \\
Zur Messung der Temperatur werden Halbleiter-Thermistoren verwendet, die einen temperaturabhängigen Widerstand besitzen. Mit der über die Thermistoren abfallenden Spannung kann die Temperatur bestimmt werden.
\\
Zum Einsatz kommt ein Thermistor im Zentrum der Konvektionszelle, dessen Höhe verstellt werden kann.
Außerdem kann die Temperatur an mehreren Stellen gleichzeitig über ein Thermistorarray an der Seite der Zelle gemessen werden. 
Das Thermistorarray besteht aus sechs Thermistoren mit einem Abstand von \SI{3}{\centi\meter}, wobei der Abstand der mittleren beiden Thermistoren \SI{4}{\centi\meter} beträgt.

