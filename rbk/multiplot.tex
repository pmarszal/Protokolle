\begin{figure}[!h]
        \centering
        \begin{subfigure}{0.5\textwidth}
        \input{./Figures/v_prof_Ra_1e3.tex}
        \caption{$\Ra = \num{e3}$}
        \label{fig:vprof_Ra_1e3}
\end{subfigure}\hfill
        \begin{subfigure}{0.5\textwidth}
        \input{./Figures/v_prof_Ra_1e4.tex}
        \caption{$\Ra = \num{e4}$}
        \label{fig:vprof_Ra_1e4}
\end{subfigure} \\
        \begin{subfigure}{0.5\textwidth}
        \input{./Figures/v_prof_Ra_1e5.tex}
        \caption{$\Ra = \num{e5}$}
        \label{fig:vprof_Ra_1e5}
\end{subfigure}\hfill
        \begin{subfigure}{0.5\textwidth}
        \input{./Figures/v_prof_Ra_1e6.tex}
        \caption{$\Ra = \num{e6}$}
        \label{fig:vprof_Ra_1e6}
\end{subfigure} \\
        \begin{subfigure}{0.5\textwidth}
        \input{./Figures/v_prof_Ra_1e7.tex}
        \caption{$\Ra = \num{e7}$}
        \label{fig:vprof_Ra_1e7}
\end{subfigure}\hfill
        \begin{subfigure}{0.5\textwidth}
                \includegraphics[width=1\textwidth]{Ra_1e7_turbolent.png}
                \caption{Snapshot für $\Ra = \num{e7}$}
                \label{fig:SimSnapshot}
         \end{subfigure}
        \caption{Geschwindigkeitsprofil numerisch berechnet bei verschiedenen Rayleigh-
        Zahlen, gemittelt über Achse durch das Zentrum der Box, gekennzeichnet durch
        schwarze Linie in~(\subref{fig:SimSnapshot}). Dabei wird lediglich jeder vierte 
        Messpunkt der Übersicht halber durch einen Kreis dargestellt.}
        \label{fig:SimVprof}
\end{figure}
